\documentclass[a4paper,12pt]{report}
\usepackage{titling}
\begin{document}
\begin{titlepage}
\title{Development of an OTA (Over the Air) Mobile Learning Telepresence Platform\\ MSc (Project/Thesis) Proposal}
\author{Kyle Galvin\\ Computer Science Department, Lakehead University\\
Jinan Fiaidhi and Sabah Mohammed}
\end{titlepage}

\maketitle

\tableofcontents

\chapter*{\centering Abstract}
Telepresence has been used in many forms in academia university for more than a decade by now as entities that help to maintain the relationship with learners and provide them with collaborative experiences without the expense of physical travel. However, the emerging technology has shifted its focus from the large class-room telepresence equipment’s to be scaled- down to mobile, wireless-networked telepresence products. With this technology shift, we are required to provide the learner with the ubiquitous ability to explore core learning contents deployed over the internet as well as to enable learners to interact with many other remote physical learning environments (e.g., Web environments, instructors, colleagues) through the use of mobile devices. This project aims at exploring this research area and to come with a solution for implementing a new type of learning objects that can be used over the air for telecollaboration and telepresence suitable for mobile platforms. 

\chapter{Lightweight Telepresence Technologies}
Microprocessors have shaped the world over the last century. Reducing in size over time at an exponential rate, we are now able to achieve things that would have been unimaginable in the past. We can squeeze more bits per volume, transport more bits and calculate more atomic operations per time than ever before. With this explosion of portability and connectivity comes a renaissance of technological growth that is unfolding before our eyes.
\section{Emerging Mobile Technologies}

\subsection{Microcontrollers \& customized System on a Chip (SoaC) components}

\begin{quotation}{
	...the BCI collects the EEG brain activity and decodes the user's intentions, which are transferred to the robot via the Internet. The robot autonomously executes the orders using the navigation system (implemented with a combination of dynamic online grid maping with scan matching, dynamic path planning, and obstacle avoidance) or the camera orientation system. Thus, the shared-control strategy is built by means  of the mental selection of robot navigation or active visual exploration task-related orders, which can be autonomously executed by the robot.}
	\begin{flushright}
		\cite{6104414}
	\end{flushright}
\end{quotation}
\begin{quote}
	Traditionally most robotic applications have involved the use of single static (non-mobile) manipulator platforms, with this technique being particularly suited to applications where the actual task is relatively well defined, the work volume is limited and safety considerations make even slightly “unexpected” motions totally unacceptable.
	\begin{flushright}
		\cite{540147}
	\end{flushright}
\end{quote}
\begin{quote}

The main objective of tele-robotics has been to develop methodologies for the control of robots at remote sites by human users at local sites. Tele-robots are suitable in certain situations such as:
\begin{enumerate}
\item The robot must operate in environments that are hazardous to human health.
\item The robot must operate at a scale that is much smaller or larger than the human size and scale.
\item The robot must operate in a location where it would be too costly for the human to be present (in terms of budget, timing requirements, and human safety).
\end{enumerate}
	\begin{flushright}
		\cite{726589}
	\end{flushright}
\end{quote}

\subsection{Mobile phones \& Cellular devices}
\cite{4469080}
\cite{6001904}
\cite{6007847}
\section{Telepresence \& Real-Time communications}
\subsection{Audio/Video compression}
\cite{4297087}
\cite{4801602}
\cite{5054795}
\subsection{Cellular network bandwidth flow \& optimization}
\cite{5710522}
\cite{1300874}
\cite{1376696}
\subsection{Privacy}
\cite{4698190}
\cite{4471983}
\cite{6270872}
\cite{1032602}
\section{Digital identification and modeling}
\subsection{Bar codes \& QR codes}
\cite{6182398}
\subsection{RFID; NFC}
\cite{5340296}
\subsection{real-time digital modeling}
\subsubsection{Stitching multiple images/videos together}
\cite{5397590}
\subsubsection{Depth matricies/maps for 3D imaging}
\subsection{Image recognition and classification}
\subsubsection{Vector Quantization}
\subsubsection{Uncertainty / Fuzzy Logic}
\subsubsection{Improved accuracy through domain-specific environments/contexts}

\bibliography{mastersthesis}
\bibliographystyle{plain}
\end{document}
